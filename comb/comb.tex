\documentclass{article}

\usepackage{microtype}

\usepackage[T1]{fontenc}
\usepackage{mathpazo}
\usepackage{amsmath,amsthm}
\newtheorem{theorem}{Theorem}
\newtheorem{proposition}{Proposition}
\newtheorem{definition}{Definition}
\newtheorem{lemma}{Lemma}
\newtheorem{exercise}{Exercise}
\newtheorem{example}{Example}
\newtheorem{corollary}{Corollary}

\usepackage{algpseudocode}
\usepackage{listings}
\lstset{basicstyle=\ttfamily}

\usepackage{tcolorbox}
\newtcolorbox{examplebox}[1]{width=\textwidth,colback={lightgray},title={#1},colbacktitle={lightgray!50},coltitle={black}}
\newtcolorbox{algobox}[1]{width=\textwidth,colback={lightgray},title={#1},colbacktitle={lightgray!50},coltitle={black}}

\usepackage{tikz}
\usetikzlibrary{shapes,snakes,positioning,decorations.pathreplacing}

\usepackage{answers}
\Newassociation{sol}{Solution}{ans}
\Newassociation{hint}{Hint}{hints}



\newcommand{\card}[1]{\ensuremath{|#1|}}
\newcommand{\nat}{\ensuremath{N}}
\newcommand{\pow}[1]{\ensuremath{2^{#1}}}
\newcommand{\comb}[2]{\ensuremath{C\!\left(#1, #2\right)}}
\newcommand{\perm}[2]{\ensuremath{P\!\left(#1, #2\right)}}
\newcommand{\fact}[1]{\ensuremath{{#1}!}}
\newcommand{\floor}[1]{\ensuremath{\left\lfloor#1\right\rfloor}}
\newcommand{\pair}[2]{\ensuremath{\left(#1, #2\right)}}

\begin{document}

\section*{Introduction}
Let us try solving some very simple problems to get a feel of
combinatorics. We have 10 candies with us and we want to distribute
them to two children A and B. How many ways are there to do this? If
we choose not to give any candy to A, then we have to give all 10
candies to B. Similarly, if we choose to give exactly one candy to A,
then we have to give 9 candies to B. The pattern here is clear. Once
we fix the number of candies that we give away to the child A, we have
no choice on the number of candies that we give to B. So the number of
ways to distribute is the same as number of ways to fix the number of
candies that we give away to A. To complete the argument, we have 11
choices on the number of candies that we give to A and therefore the
answer is 11.  In general, if we have $n$ candies and we want to
distribute all of them to two children A and B, we can do so in $n+1$
possible ways.

Now, what if there are more children? i.e., we have $n$
candies and we want to find the number of ways to distribute them to 3
children A, B and C. Unlike the 2 child version, fixing the number
of candies that we give away to A does not fix the number of candies
that we give to B or C. Suppose we give A $k$ candies. We ask how many
ways are there to distribute candies if we give exactly $k$ candies
to A. This number is exactly the number of ways to distribute the
remaining $n-k$ candies to B and C. But in the previous paragraph we
saw that this number is $n-k+1$. If we sum over all $k$ from $0$ to
$n$, we will get the required answer. Indeed,
$\sum_{k=0}^{n} (n-k+1) = 1 + \cdots + (n+1) = (n+1)(n+2)/2$. Later,
we will answer the general version of this problem where there are $n$
candies and $m$ children.

Now that we have got a taste of the kind of problems we are going to
deal with in combinatorics, we turn our attention to defining things
mathematically. First, we have to specify the kind of problems that we
are going to solve. We are interested in answering questions such as
the following

\begin{enumerate}
\item How many anagrams are there for the word 'cat'?
\item How many ways are there to form a committee of of 4 people from
  a group of 10 people?
\end{enumerate}

How can we answer these questions? A naive approach is to try and list
all possiblities. There are six anagrams for 'cat'. It would be
cumbersome to find the number of anagrams for the word 'elephant'
using this method. Also, once we have made a list of anagrams, how can
we be sure that we have listed all of them. We need a better method.

\section*{Basics}
What is common to the two problems mentioned above? In both instances,
\emph{we are interested in finding the number of elements in finite
  sets}. The set in the first question consists of the words act,
atc, cat, cta, tca and tac. Our first aim is to mathematically define
what it means for a set to have $n$ elements where $n$ is a natural
number.  Let $\nat$ denote the set of natural numbers
$\{0, 1, \ldots \}$ and let $[n]$ for any natural $n \geq 1$ denote
the set $\{0, \ldots n-1\}$. Define $[0] = \phi$.  The set $[n]$ is
the \emph{canonical} set having $n$ elements.

\begin{definition}
  Set $A$ has exactly $n$ elements, for some natural number
  $n \geq 0$, iff there is a bijection $f$ from $[n]$ to $A$.
\end{definition}

If a set $A$ has exactly $n$ elements, we denote this by
$\card{A} = n$. We say that $f$ \emph{witnesses the cardinality} of
$A$. Informally, what the above definition states is that a set has
exactly $n$ elements iff we can list the elements in the set such that
there is a first element, a second element, and so on until an
$n^{\text{th}}$ element such that every element of the set appears
exactly once in the list.

The following properties can be easily verified.

\begin{enumerate}
\item For any set $A$, there is at most one $n$ such that
  $\card{A} = n$.
\item Let $A$ and $B$ be any two sets. Then $\card{A} = \card{B} = n$
  iff there exists a bijection from $A$ to $B$.
\end{enumerate}

If $\card{A} = n$ for some $n$, then we say that $A$ is finite.

We can also compare cardinalities of sets using functions. If $f$ is
an injection (one-one function) from $A$ to $B$, then
$\card{A} \leq \card{B}$. Similarly, if $f$ is a surjection (onto
function), then $\card{A} \geq \card{B}$.

\Opensolutionfile{ans}[ans1]


\begin{exercise}
  Let \comb{n}{i} denote the number of $i$ element subsets of $[n]$.
  For any $n \geq 3$, prove that the number of 2-element subsets is at
  least the number of 1-element subsets.

  \begin{sol}
    We exhibit an injection from the number of 1-element subsets to
    the number of 2-element subsets. The function $f$ maps $\{i\}$ to
    $\{i, i+1\}$ when $i < n$ and $f$ maps to $\{n\}$ to $\{1, n\}$.
    The function $f$ is one-one because for $i \neq j$ and $i, j < n$,
    we have $\{i, i+1\} \neq \{j, j+1\}$. Also, $\{n\}$ is the only
    element that maps to $\{1, n\}$ since $n \geq 3$.
  \end{sol}
\end{exercise}


\section*{Sum and Product}
Mathematics works by building tools that can solve complex problems
from tools that can solve simpler ones. Now that we've defined the
number of elements in a set, we want to be able to argue about number
of elements in sets obtained from other sets using set-theoretic
operations (union, cross product etc.).

\begin{exercise}
  Prove the sum rule

  Let $A$ and $B$ be two finite sets such that
  $A \cap B = \phi$. Then
  $\card{A \cup B} = \card{A} + \card{B}$.

  \begin{sol}
    Let $f$ and $g$ be bijections witnessing cardinalities of $A$ and
    $B$ to be $n$ and $m$ respectively. Define the function
    $h: [n+m] \mapsto A \cup B$ as $h(i) = f(i)$ if
    $0 \leq i \leq n-1$ and $h(i) = g(i-n)$ otherwise. We now prove
    that $h$ is a bijection.

    The function $h$ is one-one. Take two elements $x$ and $y$ from
    the set $[n+m]$. If both $x$ and $y$ are in $[n]$ or in
    $R = \{n, \ldots ,n+m-1\}$, then $h(x) \neq h(y)$ because $f$ and
    $g$ are one-one. So we can assume $x \in [n]$ and $y \in R$.
    Therefore, $h(x) \in A$ and $h(y) \in B$. Since $A$ and $B$ are
    disjoint $f(x) \neq f(y)$.

    The function $h$ is onto. Take an element $y \in A \cup B$. If
    $y \in A$, then there is some element $i \in [n]$ such that
    $f(i) = y$. Therefore $h(i) = f(i) = y$. If $y \in B$, then there
    exists a $j \in [m]$ such that $g(j) = y$. Therefore
    $h(j+n) = g(j+n-n) = g(j) = y$.

    \paragraph{Note.} The above definition of $h$ and the subsequent
    proof that $h$ is a bijection may look daunting. But, the idea is
    simple. We map the first $n$ numbers to distinct elements of $A$
    as prescribed by $f$ and the next $m$ numbers to distinct elements
    of $B$ as prescribed by $g$ to construct $h$ (Here we think of $n$
    as 0, $n+1$ as 1 and so on.). This covers all elements of $A$ and
    $B$. Now since $f$ and $g$ are one-one, the only way for $h$ to
    not be one-one is if $f$ and $g$ has a common element in their
    range. But, since the ranges of $f$ and $g$ are disjoint, this is
    also ruled out. Once, we've gained enough experience with formally
    defining functions and proving that they are bijections, we will
    omit these nitty gritty details and only give intuitive
    descriptions like this.
  \end{sol}
\end{exercise}

\begin{exercise}
  Prove the product rule

  Let $A$ and $B$ be two finite sets. Then
  $\card{A \times B} = \card{A}\card{B}$.

  \begin{sol}
    Let $f$ and $g$ be bijections witnessing the cardinalities of $A$
    and $B$ to be $n$ and $m$ respectively.  Define the function
    $h:[nm] \mapsto A \times B$ as
    $h(i) = \pair{f\left(\floor{\frac{i}{m}}\right)}{g(i \mod m)}$.
    We claim that $h$ is a bijection. Indeed, for any two distinct
    elements $i$ and $j$, we have either
    $\floor{\frac{i}{m}} \neq \floor{\frac{j}{m}}$
    or $i\mod m \neq j\mod m$. Therefore, $h$ is one-one. For any
    ordered pair $\pair{a_{i}}{b_{j}}$ we have $0 \leq k_{1} \leq n-1$ and
    $0 \leq k_{2} \leq m-1$ such that
    $\pair{a_{i}}{b_{j}} = \pair{f(k_{1})}{g(k_{2})}$. Then the element
    $k = mk_1 + k_{2}$ satisfies
    $h(k) = \pair{f(k_{1})}{g(k_{2})} = \pair{a_{i}}{b_{j}}$.
  \end{sol}
\end{exercise}

We are now ready to find out how many anagrams does cat have without
enumerating all of them. Define $A_{c}$, $A_{a}$ and $A_{t}$ as the
sets of anagrams of cats starting with c, a and t respectively. How
many anagrams of cat are there that starts with c. There are two of
them -- cat and cta. (Here we are not enumerating all anagrams of
'cat', just the ones that start with the letter c. This is easier than
the original problem. So we are building a solution to a complex
problem from solutions of simpler ones. This is called a
\emph{reduction}). Therefore $\card{A_{c}} = 2$.  Similarly we have,
$\card{A_{a}} = \card{A_{t}} = 2$. If $A$ is the set of all anagrams
of cat, then we know that $A = A_{c} \cup A_{a} \cup A_{t}$ and we
know that sets $A_{c}$, $A_{a}$ and $A_{t}$ are disjoint. So $A$ must
have $2 \times 3 = 6$ elements. Later on we will solve this problem
without using enumeration at all.

Now is the time for some more definitions. Our goal is to formalize
the notion of a string.  We define $A^{n}$ where $A$ is a set and
$n \geq 1$ an integer as $A^{1} = A$ and $A^{n} = A \times A^{n-1}$.
We represent the elements of $A^{n}$ by \emph{strings} of length $n$
where each letter is from $A$ (Here $A$ is an \emph{alphabet}). For
example, the element $\pair{0}{\pair{0}{1}}$ in the set
${\{0,1\}}^{3}$ is written simply as the string $001$. Note that for
any $A$, we have $\card{A^{n}} = \card{A}^{n}$ by the produce rule.

\begin{exercise}
  Let $A = \{ a_{1}, \ldots ,a_{n} \}$. Let $\pow{A}$ denote the set
  of all subsets of $A$. Show that $\card{\pow{A}} = 2^{n}$ by
  exhibiting a bijection from ${\{ 0, 1 \}}^{n}$.

  \begin{sol}
    Let $x$ be an $n$ bit binary string. We define $h(x)$ as the
    subset $S$ of $A$ where $a_{i} \in S$ iff $x_{i} = 1$. It is easy
    to prove that $h$ is a bijection.
  \end{sol}
\end{exercise}

\begin{exercise}
  Count the number of ways to arrange 100 fruits from left-to-right
  from an infinite supply of apples and oranges.

  \begin{sol}
    Consider the set of all strings of length 100 where each letter is
    O (Orange) or A (Apple). There is a bijection from this set to the
    set of all possible arrangements. We know that there are $2^{100}$
    such strings.
  \end{sol}
\end{exercise}

\section*{Recurrences}
Can we use the above method to count the number of anagrams of
'cat'? No. The difficulty is that there is no way to express the set
of all anagrams just as the cross product of sets. However, we can
combine the union and cross product operations towards this end. We
have
$A = (\{c\}\times B_{1}) \cup (\{a\}\times B_{2}) \cup (\{t\}\times B_{3})$
where $B_{1}$, $B_{2}$ and $B_{3}$ stand for the set of all anagrams
of the words 'at', 'ct' and 'ac' respectively (If you don't get this
write down all the above sets. Enumeration may not be a good way to
solve such problems, but it is a good way to learn about them). A
little bit of thought reveals that $B_{1}$, $B_{2}$ and $B_{3}$ all
have the same number of elements, say $m$. So if we can compute $m$,
we have $\card{A} = 3m$. But, how can we compute $m$. This is where
the concept of recurrence comes in. A \emph{recurrence} describes a
function by telling how to obtain the value of the function at some
natural number $a$ from values of the function at numbers $b < a$. For
example, let us consider the function defined as $e(n) = 2e(n-1)$ for
all $n \geq 1$ and $e(0) = 1$. We can prove that $e(n) = 2^{n}$ for
all $n \geq 0$ by induction. We have $e(0) = 2^{0} = 1$. Now assume
that $e(n-1) = 2^{n-1}$. Then, we have
$e(n) = 2e(n-1) = 2.2^{n-1} = 2^{n}$, thereby proving the claim. For
simple recurrences, we can guess a closed form expression by writing
down a few examples and then try to prove that the guessed expression
is correct using induction.

\begin{exercise}
  Let $A = \{ a_{1}, \ldots ,a_{n} \}$. Find an expression for
  \begin{enumerate}
  \item the number of strings of length $k$ where each letter is from
    $A$.
  \item the number of strings of length $k$ where each letter is from
    $A$ and any element in $A$ appears at most once.
  \end{enumerate}
  in terms of $n$ and $k$.

  \begin{sol}
    \begin{enumerate}
    \item Let $f(n, k)$ denote the required number. Define $W_{a_{i}}$
      as the set of all strings of length $k$ starting with
      $a_{i}$. For $a_{i} \neq a_{j}$, we have $W_{a_{i}} \cap
      W_{a_{j}} = \phi$. The set $W_{a_{i}}$ for any $i$ has the same
      number of elements as the set of all strings of length $(k-1)$
      over the alphabet $A$. To see this, consider the bijection that
      maps $x \in W_{a_{i}}$ to the string $y$ obtained by deleting
      the first letter from $x$. The set $W_{a_{i}}$ has $f(n, k-1)$
      elements. Therefore, the required number is $\sum_{i=1}^{n} f(n,
      k-1) = nf(n, k-1)$.  We have $f(n, 1) = n$.  We have, by solving
      the recurrence, $f(n, k) = n^{k}$.

    \item We assume that $k \leq n$. Let $\perm{n}{k}$ denote the
      required number. Define $W_{a_{i}}$ as the set of all strings of
      length $k$ starting with $a_{i}$. For $a_{i} \neq a_{j}$, we
      have $W_{a_{i}} \cap W_{a_{j}} = \phi$. The set $W_{a_{i}}$ for
      any $i$ has the same number of elements as the set of all
      strings of length $(k-1)$ over the alphabet $A\setminus
      \{a_{i}\}$. Therefore, the required number is
      $n\perm{n-1}{k-1}$. We have $\perm{n}{1} = n$. We have, by
      solving the recurrence, $\perm{n}{k} = n(n-1)\ldots (n-k+1)$.
    \end{enumerate}
  \end{sol}
\end{exercise}

\begin{exercise}
  The function $f$ is defined as $f(n) = nf(n-1)$ for all $n \geq 1$
  and $f(0) = 1$. Prove that the function $f$ satisfies
  $(\floor{n/2})\log(n/2) \leq \log(f(n)) \leq n\log(n)$ for all $n
  \geq 2$ where the logarithms are to the base 2. The function $f$ is
  called the factorial function and $f(n)$ is denoted $\fact{n}$. By
  expanding the recurrence, we observe that $\fact{n} = 1 \times
  \cdots \times n$.

  \begin{sol}
    Observe that $f(n) = n(n-1)\ldots 1$. Taking logarithm to base 2
    on both sides, we have
    $\log(f(n)) = \log(n) + \log(n-1) + \cdots + \log(1) \leq
    n\log(n)$.
    Also, the first $\floor{n/2}$ terms in the
    summation are at least $\log(n/2)$ giving the lower bound on
    $\log(f(n))$.
  \end{sol}
\end{exercise}

\begin{exercise}
  Find an expression for the number of ways to arrange $n$ people in a
  row at a wedding if
  \begin{enumerate}
  \item Bride must be to the right of the groom.
  \item Bride is somewhere to the left of the groom.
  \end{enumerate}
  in terms of $n$.

  \begin{sol}
    \begin{enumerate}
    \item Let $A_{i}$ for $i$ from 1 to $(n-1)$ denote the set of all
      possible arrangements where groom is in position $i$ and bride
      is in position $i+1$. For $i \neq j$, we have
      $A_{i} \cap A_{j} = \phi$. Now $A_{i}$ for any $i$ contains
      exactly $\fact{(n-2)}$ elements since the remaining $(n-2)$
      people can be arranged in $\fact{(n-2)}$ ways. So the answer is
      $(n-1)\fact{(n-2)}$.
    \item Let $A_{i, j}$ where $1 \leq i \leq n-1$ and $j > i$ denote
      the set of all arrangements where bride is in position $i$ and
      groom is in position $j$. As before, the $A_{i,j}$s are disjoint
      and each of them has $a = \fact{(n-2)}$ elements. There are
      $b = (n-1)n/2$ such sets. So the answer is $ab$.
    \end{enumerate}
  \end{sol}  
\end{exercise}


\section*{Overcounting Helps}
Sometimes while counting a set, we will intentionally overcount and
then later correct the mistake to get the right answer. Let us look at
an example. Suppose we want to count the number of students in a class
and all we have are the lists of students enrolled for physics,
chemistry and maths. In addition, we know that each student enrolls
for exactly two out of these three subjects. Add up the number of
students in these lists. We must have counted each student twice (One
for each subject he/she enrolled for). Therefore, we can get the
number of students by dividing the total by $2$. The following set of
excercises develops this technique.

\begin{exercise}
  A function $f: A \mapsto B$, where $A$ and $B$ are finite, is called
  a $k$-to-$1$ function if for any $y \in B$, we have exactly $k$
  distinct elements from $A$ satisfying $f(x) = y$.

  Show that if there exists a $k$-to-$1$ function from $A$ to $B$, then 
  $\card{A} = k\card{B}$.

  \begin{sol}
    Let $F$ be the $k$-to-$1$ function. Let $f$ be a function
    witnessing the cardinality of $B$ to be $n$.  For each $y \in B$,
    define $A_{y}$ to be the set $k$ elements in $A$ satisfying
    $F(x) = y$. Let $g_{y} : [k] \mapsto A_{y}$ be a bijection
    witnessing the cardinality of $A_{y}$ to be $k$. Define the
    bijection $h : [kn] \mapsto A$ as $h(i) = g_{b_{f(m)}}(i\mod k)$,
    where $m = \floor{\frac{i}{k}}$ and
    $B = \{b_{0}, \ldots ,b_{\card{B}-1}\}$. The function $h$ simply
    maps first $k$ numbers to the preimages of $b_{0}$, next $k$
    numbers to the preimages of $b_{1}$ and so on.

    (An alternate proof) Observe that $A_{y}$ and $A_{x}$ are disjoint
    for $x \neq y$. And, each $A_{x}$ has exactly $k$
    elements. Furthermore, we have $A = \bigcup_{y\in B} A_{y}$. By
    applying the sum rule, we have $\card{A} = k\card{B}$.
  \end{sol}
\end{exercise}

\begin{exercise}
  Find an expression for the number of distinct diagonals for an
  $n$-sided polygon.

  \begin{sol}
    Let the vertices of the polygon be labelled $0, \ldots ,n-1$ such
    $i$ is adjacent to $i+1$ modulo $n$. Consider the set
    $A = \{ \pair{i}{j} : j \not\in \{ i, i+1, i-1 \}\mod n \}$.
    There are $n(n-3)$ elements in $A$. There is a natural $2$-to-$1$
    mapping from $A$ to the set of diagonals where $\pair{i}{j}$ and
    $\pair{j}{i}$ map to the diagonal between vertices $i$ and $j$. The
    answer is $n(n-3)/2$.
  \end{sol}
\end{exercise}

\begin{exercise}
  Find an expression for the number of ways to place two
  non-threatening rooks in a $k$-by-$k$ chessboard in terms of $k$.

  \begin{sol}
    Consider the set
    $A = \{ \pair{\pair{i}{j}}{\pair{i'}{j'}} : i' \neq i, j' \neq j
    \}$
    where all numbers are between 0 and $k-1$.  The elements of $A$
    represent pairs of non-threatening positions for two rooks on a
    $k$-by-$k$ chessboard. There are $k^{2}(k^{2} - 2k + 1)$ elements
    in $A$. But elements $\pair{\pair{i}{j}}{\pair{i'}{j'}}$ and
    $\pair{\pair{i'}{j'}}{\pair{i}{j}}$ in $A$ correspond to the same
    placement of two rooks and these are the only pairs that
    correspond to this placement. So there is a 2-to-1 function from
    $A$ to the set of all possible placements of two rooks. So the
    answer is $k^{2}(k^{2} - 2k + 1)/2$.
  \end{sol}
\end{exercise}

\begin{exercise}
  \begin{enumerate}
  \item Find an expression for the number of ways to distribute $n$ labelled
    balls into $k$ labelled bins in terms of $n$ and $k$.
    
  \item Find an expression for the number of ways to distribute $n$ labelled
    balls into $k$ unlabelled bins in terms of $n$ and $k$.
  \end{enumerate}
  
  \begin{sol}
    \begin{enumerate}
    \item Consider the set of all strings $A$ of length $n$ where each
      letter is from the set $\{1, \ldots , k\}$. There is a bijection
      from this set to the set of all ways to distribute $n$ labelled
      balls into $k$ labelled bins. i.e., $i^{\text{th}}$ ball is
      placed in $j^{\text{th}}$ bin iff the $i^{\text{th}}$ letter of
      the string is $j$. So the answer is $k^{n}$.

    \item We show that there is a $\fact{k}$-to-1 mapping from $A$ to
      the set of all ways to distribute $n$ labelled balls into $k$
      unlabelled bins. Indeed, given any element in $A$ we can permute
      the alphabet in exactly $\fact{k}$ ways without changing the
      distribution. So the answer is $k^{n}/\fact{k}$.
    \end{enumerate}
  \end{sol}
\end{exercise}

\begin{exercise}
  Let $A = \{ a_{1}, \ldots ,a_{n} \}$. Find an expression for the
  number of strings where each letter in the string is an element of
  $A$ and the string has $k_{i}$ occurrences of $a_{i}$ for each $i$
  (in terms of $n$ and $k_{i}$s).

  \begin{sol}
    Let $m = \sum_{i = 1}^{n} k_{i}$ denote the length of the
    strings. Suppose that all letters are distinct, say
    $\{a_{1,1}, \ldots ,a_{1,k_{1}}, \ldots ,a_{n,k_{n}} \}$. Let
    $A_{0}$ be the set of all strings over this new alphabet. Clearly,
    we have $\card{A_{0}} = \fact{m}$.  Now assume that the letters
    $a_{1,1}, \ldots ,a_{1,k_{1}}$ are replace by $a_{1}$ and let
    $A_{1}$ be the set of all strings over this alphabet where $a_{1}$
    appears exactly $k_{1}$ times and every other letter appears
    exactly once. We exhibit a $\fact{k_{1}}$-to-1 mapping from
    $A_{0}$ to $A_{1}$. This mapping $h$ maps a string in $A_{0}$ into
    the string obtained by replacing $a_{1, i}$ for all
    $1 \leq i \leq k_{1}$ by $a_{1}$. Let $x \in A_{0}$ and let
    $I = \{i_{1}, \ldots ,i_{k_{1}}\}$ be the positions of
    $a_{1, 1}, \ldots ,a_{1,k_{1}}$ and let $x' = h(x)$. Then, any and
    only those strings in $A_{0}$ where the letters
    $a_{1, 1}, \ldots ,a_{1,k_{1}}$ occur in positions $I$ and all
    other letters are in their same position as in $x$ will map to
    $x'$. There are exactly $\fact{k_{1}}$ ways to arrange
    $a_{1, 1}, \ldots ,a_{1,k_{1}}$ in positions $I$. Therefore, we
    have $\card{A_{1}} = \card{A_{0}}/\fact{k_{1}}$.  Similarly, we
    construct $A_{i+1}$ from $A_{i}$ by replacing letters $a_{i+1, j}$
    for $j$ from $1$ to $k_{i+1}$ with $a_{i+1}$. It holds that
    $\card{A_{i+1}} = \card{A_{i}}/\fact{k_{i+1}}$ for all $i$ from 1
    to $n-1$. It follows that
    $\card{A_{n}} = \frac{\fact{m}}{\fact{k_{1}}\ldots\fact{k_{n}}}$.
  \end{sol}
\end{exercise}

\begin{exercise}
  Let $A = \{ a_{1}, \ldots ,a_{n} \}$. Find an expression for the
  number of $k$-element subsets of $A$ in terms of $n$ and $k$.

  \begin{sol}
    There exists a bijection from strings in ${\{0, 1\}}^{n}$ with
    exactly $k$ 1's to the $k$-element subsets of $A$. And, we know
    that the number of strings with exactly $k$ 1's and $(n-k)$ 0's is
    $\frac{\fact{n}}{\fact{k}\fact{(n-k)}}$. This number is denoted
    $\comb{n}{k}$.
  \end{sol}
\end{exercise}

\begin{exercise}
  Find an expression for

  \begin{enumerate}
  \item the number of ways to split $2n$ people into 2
  teams of $n$ people each.
  \item the number of ways to split $2n+1$ people into 2
  teams of $n+1$ and $n$ people.
\end{enumerate}

  in terms of $n$.

  \begin{sol}
    \begin{enumerate}
    \item Label the players $1, \ldots ,2n$ with $0$ (for team $0$) or
      $1$ (for team $1$). The number of strings with exactly $n$ 1's
      and $n$ 0's is $\frac{\fact{(2n)}}{\fact{n}\fact{n}}$. But,
      changing the team labels does not change the split into teams
      and there are exactly 2 different ways to label teams given a
      split. So the answer is
      $\frac{\fact{(2n)}}{2(\fact{n})(\fact{n})}$.

    \item Here we just have to count the number of strings with $n$
      0's and $(n+1)$ 1's. The answer is
      $\frac{\fact{(2n+1)}}{\fact{(n+1)}\fact{n}}$. If we define $h$ as
      the function mapping such a string to the split where
      $i^{\text{th}}$ player is in team $0$ iff $i^{\text{th}}$ bit is
      $0$, then $h$ is a bijection from such strings to all possible
      splits.
    \end{enumerate}
  \end{sol}  
\end{exercise}

\begin{exercise}
  Find an expression for the number of integral solutions to
  $x_{1} + \ldots + x_{k} = n$ where $x_{i} \geq 0$ for all
  $1 \leq i \leq k$ in terms of $n$ and $k$.

  \begin{sol}
    We exhibit a bijection from all possible strings with $(k-1)$ 1's
    and $n$ 0's to the set of all solutions. We let $x_{i}$ to be the
    number of $0$'s between the ${(i-1)}^{\text{st}}$ and the
    $i^{\text{th}}$ 1 in the string for $2 \leq i \leq k-1$. We let
    $x_{1}$ to be the number of 0's before the first 1 and we let
    $x_{k}$ to be the number of 0's after the last 1. It is easy to
    see that this is a bijection. The answer is \comb{n+k-1}{n}.
  \end{sol}
\end{exercise}

\section*{Pitfall: Overcounting}

Let us try to count the number of 4-bit strings with at least 2 zeroes
in it. There are $\comb{4}{2} = 6$ ways to choose 2 positions for the
2 mandatory zeroes and for each of these 6 choices, the other two bits
could take on all $4$ possible values. So the answer must be
$6\times 4 = 24$. But, there are only 16 4-bit strings! Where did we
go wrong? Let us look into how our solution applied the methods that
we learned in more detail. First, we split our solution into 6 sets
$S_{ij}$ where $i$ and $j$ are distinct numbers from 1 to 4 where
$i < j$. Here $S_{12}$ contains all 4-bit strings where there are
zeroes in positions 1 and 2. Therefore,
$S_{12} = \{ 0000, 0001, 0010, 0011 \}$. Then we argued that by
symmetry any $S_{ij}$ must have 4 elements as well. Then we applied
the sum rule to obtain our answer. But
$S_{23} = \{ 0000, 0001, 1000, 1001 \}$. Here $S_{12}$ and $S_{23}$
are not disjoint and the sum rule cannot be applied.

Even though the sum rule turned out to be inapplicable in the previous
solution, we can solve the same problem using the sum rule itself. The
trick is to partition the set of all strings with at least 2 zeroes
into disjoint sets. Define $S_{i}$ to be the set of all 4-bit strings
with \emph{exactly} $i$ zeroes. Then the set we are trying to count is
$S_{2}\cup S_{3}\cup S_{4}$. Clearly, these sets are disjoint and sum
rule is applicable. We have $\card{S_{i}} = \comb{4}{i}$. And, the
required answer is
$\comb{4}{2} + \comb{4}{3} + \comb{4}{4} = 6 + 4 + 1 = 13$.

\section*{An Application: Proving Combinatorial Identities}
Double counting is a technique that uses counting to prove
combinatorial identities. We will illustrate this technique by proving
that $\comb{n}{k} = \comb{n}{n-k}$ for all $0 \leq k \leq n$ and
$n \geq 1$. Let $A$ be the set of all $k$-element subsets of the set
$[n]$. Then $\card{A} = \comb{n}{k}$ by definition. We can also obtain
a set in $A$ by choosing $n-k$ elements in $[n]$ to be excluded from
our set. There are $\comb{n}{n-k}$ ways to choose the $n-k$ elements
in $[n]$ to be excluded from our set and each of these ways gives a
distinct $k$-element subset of $[n]$. So $\card{A} = \comb{n}{n-k}$.

\begin{exercise}
  Prove the following identities using double counting.
  \begin{enumerate}
  \item $\comb{n}{k} = \comb{n-1}{k-1} + \comb{n-1}{k}$ for
    $1 \leq k \leq n$. Define $\comb{n}{0} = 1$ and $\comb{n}{k} = 0$
    when $n < k$.
  \item ${(x+y)}^{n} = \sum_{k=0}^{n} \comb{n}{k}{x}^{k}{y}^{n-k}$
  \item Explain why Pascal's triangle works.
  \item $\sum_{k = 0}^{n} \comb{n}{k} = 2^n$
  \item $\sum_{k = 0}^{n} {(-1)}^{k}\comb{n}{k} = 0$
  \item $\comb{n}{r}\comb{n-r}{k} = \comb{n}{r+k}\comb{r+k}{r}$ for
    $r+k \leq n$ and $n$, $k$ and $r$ are positive integers.
  \item $\fact{k}\comb{n}{k} = \perm{n}{k}$
  \end{enumerate}
\end{exercise}

\section*{The Inclusion-Exclusion Principle (PIE)}
Let $A$ and $B$ be two finite sets. We have the following identity.
\begin{equation}
  \card{A\cup B} = \card{A} + \card{B} - \card{A\cap B}
\end{equation}
How can we prove this identity? Look at the expression on the right
hand side.  The term \card{A} counts each element in $A\setminus B$
once and the term \card{B} counts each element in $B\setminus A$
once. What about elements in $A\cap B$? Each element in this part of
$A\cup B$ is counted twice, once in \card{A} and once in \card{B}. So
we subtract $\card{A\cap B}$ to eliminate this overcounting.

PIE is a generalization of the above formula. Let
$A_{1}, \ldots ,A_{n}$ be $n$ finite sets. We use $A_{J}$ where
$J\subseteq \{1, \ldots ,n\}$ to denote intersection of all $A_{i}$s
where $i\in J$.
\begin{equation}
  \card{A_{1}\cup \ldots \cup A_{n}} = \sum_{i = 1}^{n} \sum_{J: \card{J} = i}{(-1)}^{i+1} A_{J}
\end{equation}

\Closesolutionfile{ans}
\section*{Answers}
\input{ans1}

\end{document}

%%% Local Variables:
%%% mode: latex
%%% TeX-master: t
%%% End:

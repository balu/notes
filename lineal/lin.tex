\documentclass{article}

\usepackage{microtype}

\usepackage[T1]{fontenc}
\usepackage{mathpazo}
\usepackage{amsmath,amsthm}
\newtheorem{theorem}{Theorem}
\newtheorem{proposition}{Proposition}
\newtheorem{definition}{Definition}
\newtheorem{lemma}{Lemma}
\newtheorem{exercise}{Exercise}
\newtheorem{example}{Example}
\newtheorem{corollary}{Corollary}

\usepackage{algpseudocode}
\usepackage{listings}
\lstset{basicstyle=\ttfamily}

\usepackage{tcolorbox}
\newtcolorbox{examplebox}[1]{width=\textwidth,colback={lightgray},title={#1},colbacktitle={lightgray!50},coltitle={black}}
\newtcolorbox{algobox}[1]{width=\textwidth,colback={lightgray},title={#1},colbacktitle={lightgray!50},coltitle={black}}

\usepackage{tikz}
\usetikzlibrary{shapes,snakes,positioning,decorations.pathreplacing}

\usepackage{answers}
\Newassociation{sol}{Solution}{ans}
\Newassociation{hint}{Hint}{hints}



\DeclareMathOperator*{\vdim}{dim}
\DeclareMathOperator*{\vspan}{span}
\DeclareMathOperator*{\vrank}{rank}
\DeclareMathOperator*{\trank}{rank}
\DeclareMathOperator*{\tnull}{null}
\DeclareMathOperator*{\tdim}{dim}
\DeclareMathOperator*{\tnullity}{nullity}
\DeclareMathOperator*{\trange}{range}

\newcommand{\swap}{\leftrightarrow}

\begin{document}

\Opensolutionfile{ans}[ans1]

\begin{example}
  The set of all real numbers along with addition and multiplication
  has a number of nice properties. The set is closed under both
  operations. Both operations are commutative, associative, has an
  identity, and has inverses (except for $0$ which doesn't have a
  multiplicative inverse). Also, multiplication distributes over
  addition. Any set that has all these properties is called a field.
\end{example}

\begin{example}
  A field can contain a finite number of elements as well. The
  smallest field called $F_2$ has just two elements $0$ and $1$.
\end{example}

\begin{example}
  Consider the Euclidean plane $R^2$. We define addition of two
  vectors as coordinate-wise addition,
  $(x_1, y_1) + (x_2, y_2) = (x_1 + x_2, y_1 + y_2)$, and multiplying
  a vector by a scalar as $c(x, y) = (cx, cy)$.
\end{example}

\section{Basis, Dimension, and Linear Independence}

\begin{example}
  Consider the vector space $R^2$. Any vector $(x, y)$ can be
  expressed as the linear combination $x(1, 0) + y(0, 1)$. We say that
  the vectors $(1, 0)$ and $(0, 1)$ span the whole space $R^2$. These
  are not the only vectors that span $R^2$. The vectors $(1, 1)$ and
  $(1, 2)$ also span $R^2$. Because, we can express any vector
  $(x, y) = (2x-y)(1, 1) + (y-x)(1, 2)$.
\end{example}


\begin{example}
  If we consider the three vectors $(1, 0)$, $(0, 1)$, and $(1, 1)$,
  they span the whole space $R^2$. However, only two of these vectors
  are required to span $R^2$. This is because, for example, $(1, 1)$
  can be expressed as the linear combination $(1, 0) + (0, 1)$ of the
  other two vectors. Therefore, any linear combination of these three
  vectors can be converted to a linear combination of the first two
  vectors. Indeed,
  $\alpha_1 (1, 0) + \alpha_2 (0, 1) + \alpha_3 (1,1) = (\alpha_1 +
  \alpha_3) (1, 0) + (\alpha_2 + \alpha_3) (0, 1)$ for any
  $\alpha_1, \alpha_2, \alpha_3$.
\end{example}

Linearly independent set of vectors that span the whole vector space
is a basis.

\begin{exercise}
  Show that the set $\{(1, 0), (a, b)\}$ is a basis for the vector
  space $R^2$ if and only if $(a, b)\neq \alpha(1, 0)$ for any
  $\alpha$.
\end{exercise}

\begin{theorem}
  If $B$ and $B'$ are bases for the vector space $V$, then
  $|B| = |B'|$.
\end{theorem}
\begin{proof}
  We are going to use proof by contradiction. Let
  $B = \{u_1,\dotsc ,u_n\}$ and $B' = \{v_1,\dotsc ,v_m\}$ such that
  $n\leq m$. We are going to construct sequences $S_0, \dotsc, S_n$
  such that $\vspan(S_i) = V$ for all $i$, $S_0$ contains only vectors
  from $B$, and $S_n$ contains only vectors from $B'$. Start with
  $S_0 = (u_1, \dotsc, u_n)$. Now consider the sequence
  $S = (u_1, \dotsc, u_n, v_1)$. Since $B$ is a basis, we can express
  $v_1$ as the linear combination of the previous vectors. Since $B'$
  is also a basis, we have $v_1 \neq 0$ and therefore at least one of
  the $u_i$s must have a non-zero coefficient in this linear
  combination. Choose the minimum such $i$. We can remove $u_i$ from
  $S$ to get the sequence $S_1$. Since $u_i$ can be expressed as a
  linear combination of $S_1$ and $v_1$ is a linear combination of
  $u_1,\dotsc,u_n$, we have $\vspan(S_0) = \vspan(S_1)$. In the
  $i^\text{th}$ step, we have $S_i$ that contains some vectors from
  $B$ in the left part and $v_1,\dotsc,v_i$ as its rightmost
  elements. Now consider $S = S_i, v_{i+1}$. We can express $v_{i+1}$
  as a linear combination of vectors in $S_i$ because
  $\vspan(S_i) = \vspan(B)$. This linear combination must contain
  atleast one $u_i$ that has a non-zero coefficient because $B'$ is
  linearly independent. As before, we can remove $u_i$ from $S$ to
  obtain $S_i$ and satisfy $\vspan(S_i) = \vspan(S_{i+1})$. The
  sequence $S_n$ is $(v_1,\dotsc,v_n)$ and satisfies
  $\vspan(S_n) = V$. Now if $m > n$, then the vector $v_{n+1} \in V$
  can be expressed as a linear combination of $v_1,\dotsc,v_n$. This
  is a contradiction to the fact that $B'$ is linearly independent.
\end{proof}

\begin{corollary}
  If $S$ is linearly independent and has $\vdim(V)$ vectors, then
  $S$ is a basis of $V$.
\end{corollary}
\begin{proof}
  Let $B$ be any basis of $V$. Apply the previous proof with $B' =
  S$. This will prove that $\vspan(S) = V$.
\end{proof}

\begin{proposition}
  Any $S$ such that $\vspan(S) = V$ contains a basis of $V$.
\end{proposition}
\begin{proof}
  If $0\in S$, we can remove $0$ without changing the span.  If $S$ is
  linearly dependent, then choose a $v$ that can be expressed as a
  non-zero linear combination of other vectors and remove $v$. This
  does not change the span. Continue until the set of vectors are
  linearly independent.
\end{proof}

\begin{proposition}
  Any linearly independent set $S\subseteq V$ can be extended to be a
  basis of $V$.
\end{proposition}
\begin{proof}
  Choose $v\in V\setminus \vspan(S)$ and add it to $S$. This set is
  linearly independent and spans a larger vector space. Continue until
  the span equals $V$.
\end{proof}

Ordered basis and coordinate representation of vectors.

\section{Subspaces}

\begin{exercise}
  A set of points $L\subseteq R^2$ is called a \emph{line} if there
  are real numbers $a$, $b$, and $c$ such that all points
  $(x, y)\in L$ satisfy the equation $ax + by = c$.

  When is a line a subspace of $R^2$?

  \begin{sol}
    When $(0, 0)$ satisfies the equation for the line.
  \end{sol}
\end{exercise}

\begin{exercise}
  Show that the only subspaces of $R^2$ are the trivial subspaces $0$
  and $R^2$ and lines passing through the origin.

  \begin{sol}
    Any non-trivial subspace $L$ has dimension 1. Therefore, we can
    write
    $L = \{(x, y) \in R^2: (x, y) = a(x_0, y_0) \text{ for some } a\in
    R\}$ for some $(x_0, y_0)\in R^2$. The set $L$ is the line
    represented by the equation $y_0 x - x_0 y = 0$. Any point in $L$
    satisfies this equation. On the other hand, if $(a, b)$ satisfies
    this equation, then $a y_0 = b x_0$ or $(a, b) = b/y_0(x_0, y_0)$
    when $y_0 \neq 0$. If $y_0 = 0$, then $x_0\neq 0$ since $L$ is a
    non-trivial subspace. Then, we have $b = 0$ and
    $(a, 0) = a/x_0(x_0, 0)$.
  \end{sol}
\end{exercise}

\begin{exercise}
  Can you describe the subspaces of ${(F^p)}^2$ in a similar fashion
  to the subspaces of $R^2$.
\end{exercise}

\begin{exercise}
  Describe all the subspaces of $R^3$.
\end{exercise}

\section{Systems of Linear Equations}

We are interested in the solutions of the homogenous system $Ax =
0$. When does this system have a non-trivial solution? Can we describe
all solutions of the system?

Matrix Multiplication: The element-wise, row-wise, and column-wise
views.

Elementary row operations on a matrix $A$ are:

\begin{description}
\item [$r_i\swap r_j$] Swap two rows of $A$.
\item [$r_i\gets c r_i$] Multiply any row of $A$ with an $\alpha\neq 0$.
\item [$r_i \gets r_i + c r_j$] Add a scalar multiple of a row to
  another row.
\end{description}

Two matrices are row-equivalent if one can be obtained from the other
using elementary row operations.

\begin{theorem}
  Row-equivalence is an equivalence relation.
\end{theorem}
\begin{proof}
  Antisymmetry can be proved by observing that all row operations are
  invertible. Reflexivity and transitivity are then trivial.
\end{proof}

Describe row-reduced echelon form and elementary matrices.

Matrix Inverse: Computation using Gaussian-Elimination and
${(AB)}^{-1} = B^{-1}A^{-1}$

\begin{theorem}
  Elementary row operations do not change the row space of the matrix.
\end{theorem}
\begin{proof}
  Let $r_1,\dotsc,r_m$ be the rows of the matrix. We will show how to
  transform between linear combinations ($\alpha$) of the rows before
  the operation to linear combinations ($\alpha'$) of the rows after
  the operation to generate the same vector.
  \begin{description}
  \item [$r_i \swap r_j$] Swap the coefficients of $r_i$ and $r_j$.
  \item [$r_i \gets c r_i$] Use $\alpha_i' = c^{-1}\alpha_i$ and
    $\alpha_i = c\alpha_i'$.
  \item [$r_i \gets r_i + c r_j$] Use $\alpha_j' = \alpha_j - c\alpha_i$
    and $\alpha_j = \alpha_j' + c\alpha_i'$.
  \end{description}
\end{proof}

More interesting is how the column space changes with row
operations. Consider the following example,

\begin{example}
  \begin{equation*}
    A =
    \begin{pmatrix}
      1 & 2\\
      1 & 2
    \end{pmatrix}
    \to
    \begin{pmatrix}
      1 & 2\\
      0 & 0
    \end{pmatrix}
    = B
  \end{equation*}
  The column space of $A$ is $\{[v; v] : v\in R\}$ and that of $B$ is
  $\{[v; 0] : v\in R\}$. Therefore, the column space may change as a
  result of a row operation. But, the dimension of the column space
  was unchanged by the above operation. Infact, we can prove something
  stronger. In the above example, columns 1 and 2 of $A$ were linearly
  dependent and remained so after the row operation. On the other
  hand, a set of columns that are linearly independent will also
  remain so after any elementary row operation, Here, we can see that
  columns 1 and 2 remained non-zero after the operation.
\end{example}

We now formally prove the above observation.

\begin{lemma}
  Let $U$ be a subset of columns of a matrix. Let $U'$ be the
  corresponding set of columns after an elementary row operation, then
  a linear combination $\alpha$ witnesses the linear dependence of $U$
  if and only if it witnesses the linear dependence of $U'$.
\end{lemma}
\begin{proof}
  Let $u_1,\ldots,u_n$ be the columns of the $m\times n$ matrix where
  $u_i = [a_{1i}; \dotso ; a_{mi}]$. Consider any
  $U\subset\{u_1,\dotsc,u_n\}$. Let $U'$ be the vectors in $U$ after
  performing the elementary row operation.
  
  \begin{description}
  \item [$r_i \swap r_j$] The linear combinations in row $i$ and $j$
    are simply swapped. Since both are $0$, the result follows.
    
  \item [$r_i \gets c r_i$] The linear combination at row $i$ is
    multiplied by $c$ or $c^{-1}$ which does not change the value $0$.
    
  \item [$r_i \gets r_i + c r_j$] Assume
    $\alpha_1 u_1 + \dotsm + \alpha_k u_k = 0$. This implies
    $\alpha_1 a_{i1} + \dotsm \alpha_k a_{ik} = 0$ and
    $\alpha_1 a_{j1} + \dotsm \alpha_k a_{jk} = 0$. Then the row $i$
    of $\alpha_1 u_1' + \dotsm + \alpha_k u_k'$ is equal to
    $\alpha_1 (a_{i1} + c a_{j1}) + \dotsm \alpha_k (a_{ik} + c
    a_{kj}) = 0$. For the other direction, consider
    $\alpha_1 u_1' + \dotsm + \alpha_k u_k' = 0$, then we have
    $\alpha_1 (a_{i1} + c a_{j1}) + \dotsm \alpha_k (a_{ik} + a_{jk})
    = 0$ and $\alpha_1 a_{j1} + \dotsm \alpha_k a_{jk} = 0$. These two
    equalities imply $\alpha_1 a_{i1} + \dotsm \alpha_k a_{ik} = 0$.
  \end{description}
\end{proof}

\begin{corollary}
  Elementary row operations do not change the dimension of the column
  space of the matrix.
\end{corollary}
\begin{proof}
  Consider a subset of columns that form a basis for the column
  space. These columns will remain linearly independent after
  performing elementary row operations on the matrix. So the dimension
  of the new column space is atleast that of the original. Since we
  can also obtain the original matrix from the new one using
  elementary row operations, it follows that the dimension of the
  original column space is atleast that of the new one.
\end{proof}

\begin{theorem}
  The row rank and column rank of a matrix are the same.
\end{theorem}
\begin{proof}
  Let $A$ be an $m\times n$ matrix. Consider the matrix $R$ that is
  the RREF of $A$. Let $r$ be the number of non-zero rows in $R$. Then
  row rank of $A$ is $r$ since row space of $A$ and row space of $R$
  are the same and the rows of $R$ are linearly independent. Let $U$
  be the column space of $A$ and let $U'$ be the column space of
  $R$. We have $c = \vdim(U) = \vdim(U')\leq r$. Considering the
  transpose of $A$ and the same argument, we have $r\leq c$ as well.
\end{proof}

\begin{theorem}
  If $R$ is obtained from $A$ using elementary row opertions, then
  $Ax = 0$ and $Rx = 0$ has the same solutions.
\end{theorem}
\begin{proof}
  $0$ is a common solution. Any non-zero solution $x*$ to $Ax = 0$ can
  be viewed as witnessing linear dependence of columns of
  $A$. Therefore, $x*$ also witnesses linear dependence of columns of
  $R$ or in other words, $Rx* = 0$. The reverse is also true by the
  same argument.
\end{proof}

Consider the system $Ax = b$ for $b\neq 0$. Describe the solutions of
this system.

\begin{theorem}
  Let $A$ and $A'$ be $m\times n$ matrices such that $A'$ is obtained
  from $A$ using an elementary row operation. Let $b$ be an
  $m\times 1$ vector and let $b'$ be the vector obtained from $b$ by
  applying the same row operation. Then $A\alpha = b$ if and only if
  $A'\alpha = b'$
\end{theorem}
\begin{proof}
  We split the proof based on the row operation used to obtain $A'$
  from $A$.
  \begin{description}
  \item [$r_i\swap r_j$] The linear combinations on row $i$ and row
    $j$ are swapped which means the corresponding rows of $b$ (or
    $b'$) are swapped resulting in $b'$ (respectively $b$).
  \item [$r_i\gets cr_i$] If $\alpha$ is a solution to $Ax = b$, then
    row $i$ of $A'\alpha$ is row $i$ of $A\alpha = b$ multiplied by
    $c$, i.e., same as that of $b'$. If $\alpha$ is a solution to
    $A'x = b'$, then row $i$ of $A\alpha$ is row $i$ of $b'$
    multiplied by $c^{-1}$, i.e., same as that of $b$.
  \item [$r_i\gets r_i + c r_j$] Let $A\alpha = b$, then row $i$ of
    $A'\alpha$ is $b_i + c b_j = b_i'$. If $A'\alpha = b'$, then we
    have
    $\alpha_1 (a_{i1} + c a_{j1}) + \dotsm + \alpha_n (a_{in} + c
    a_{jn}) = b_i' = b_i + c b_j$ and
    $\alpha_1 a_{j1} + \dotsm + \alpha_n a_{jn} = b_j' = b_j$. Then,
    row $i$ of $A\alpha$ is
    $\alpha_1 a_{i1} + \dotsm + \alpha_n a_{in} = b_i' - c b_j = b_i$.
  \end{description}
\end{proof}

The above theorem implies the following procedure to solve the linear
system $Ax = b$ when $b\neq 0$. Let $R$ be the RREF of $A$ and let
$b'$ be the vector obtained from $b$ by applying the same sequence of
elementary operations. Then, the solutions of $Rx = b'$ are the same
as that of $Ax = b$. If $b'$ has a non-zero row $i$ and row $i$ of $R$
is zero, then the system has no solution. Otherwise, let $r$ be the
rank of $R$. Then we have $i_1 < \dotsm < i_r \leq n$ and the systems
of equations are:

\begin{align*}
  x_{i_1} &+ \sum_{j > i_1, j\not\in\{i_1\dotsc,i_r\}} a_{1j} x_j = b_1\\
          &\vdots&\\
  x_{i_r} &+ \sum_{j > i_r, j\not\in\{i_1\dotsc,i_r\}} a_{1j} x_j = b_r
\end{align*}

Now we can substitute any value for $n-r$ variables other than
$x_{i_j}$ and obtain unique values for $x_{i_1},\dotsc,x_{i_r}$ that
is a solution. If $r = n$, we get a unique solution. If $r < n$, then
the set of solutions is in one-to-one correspondence with a vector
space of dimension $n-r$. If the field is finite, this means that the
system will have exactly ${|F|}^{n-r}$ solutions.

\begin{enumerate}
\item $\vrank(A) = \vrank(A|b) < n$: Many solutions.
\item $\vrank(A) = \vrank(A|b) = n$: Unique solution.
\item $\vrank(A) < \vrank(A|b)$: No solutions.
\end{enumerate}

\section{Linear Transformations}

A mapping that preserves structure is a homomorphism. A bijective
homomorphism is an isomorphism.

A function $T$ from a vector space $V$ over $F$ to $W$ over $F$ is
called a linear transformation if it is a homomorphism. In other
words, $T(\alpha u + v) = \alpha T(u) + T(v)$ for all $u, v\in V$ and
all $\alpha\in F$.

\begin{lemma}
  If $T : U\mapsto V$ is an isomorphism between vector spaces, then
  $\vdim(U) = \vdim(V)$.
\end{lemma}
\begin{proof}
  Let $u_1, \dotsc, u_n$ be a basis for $U$. Let $v_i = T(u_i)$ for
  all $i$. Then, the vectors $v_1,\dotsc,v_n$ form a basis for $V$.
  First of all, they are linearly independent because if
  $\sum_i \alpha_i v_i = 0$, then $T(\sum_i \alpha_i u_i) = 0$. Since
  $T$ is an isomorphism, only the zero vector can map to $0$. Take any
  vector $v\in V$. Let $\sum_i\alpha_iu_i = u\in U$ such that
  $T(u) = v$. Then, we have $T(u) = \sum_i\alpha_iv_i = v$ proving
  that $v_1,\dotsc,v_n$ spans $V$.
\end{proof}

\begin{lemma}
  If $U$ and $V$ are vector spaces of the same dimension, then they
  are isomorphic.
\end{lemma}
\begin{proof}
  Consider the linear transformation $T(u_i) = v_i$, where $u_i$s and
  $v_i$s are bases of $U$ and $V$ respectively. This function is
  one-one because $T(\sum_i\alpha_i u_i) = T(\sum_i\beta_i u_i)$
  implies $\sum_i\alpha_i v_i = \sum_i\beta_i v_i$ which implies
  $\alpha = \beta$ because $v_i$s are independent. This function is
  onto because $v = \sum_i \alpha_i v_i = T(\sum_i\alpha_i u_i)$.
\end{proof}


\begin{align*}
\tnull(T) &= \{x : T(x) = 0 \}\\
\trange(T) &= \{y : \text{there exists an $x$ such that $T(x) = y$}\}
\end{align*}

Prove that $\tnull(T)$ and $\trange(T)$ are subspaces of $V$ and $W$
respectively.

\begin{example}
  Consider the linear transformation $T(x, y) = (x, x)$. In
  particular, we have $T(1, 0) = (1, 1)$ and $T(0, 1) = (0,
  0)$. Consider the following matrix

  \begin{equation*}
    A = 
    \begin{pmatrix}
      1 & 0\\
      1 & 0
    \end{pmatrix}
  \end{equation*}

  We have $A[x; y] = [x; x]$. Therefore, $T(v) = Av$ for any vector
  $v\in R^2$. We say that $A$ represents the linear transformation $T$
  with respect to the standard bases of the domain and range of $T$.
\end{example}

\begin{theorem}
  Let $T$ be a linear transformation from $F^n$ to $F^m$. There is an
  $m\times n$ matrix $A$ such that $T(x) = Ax$.
\end{theorem}
\begin{proof}
  The columns of $A$ are the coordinates, with respect to standard
  basis of $F^m$, of images of the standard basis vectors of $F^n$.
\end{proof}

Ofcourse, there is nothing special about the standard bases in
Theorem~\ref{}. Let $T : U\mapsto V$ be a linear transformation. If
the vectors in $U$ are represented as coordinates over $B$ and vectors
in $V$ as coordinates over $B'$, we can choose the columns of $A$ as
the coordinates, with respect to $B'$, of images of vectors in $B$.

The subspace $\trange(T)$ is the column space of $A$.

The null-space of a matrix $A$ is $\{ x : Ax = 0 \}$.

\begin{lemma}
  There is an isomorphism between the null-space of $A$ and $A'$ when
  $A$ and $A'$ are column equivalent.
\end{lemma}
\begin{proof}
  The isomorphism depends on the type of column operation:

  \begin{description}
  \item [$c_i \swap c_j$] Swap the coordinates at $i$ and $j$.
  \item [$c_i \gets \beta c_i$] Multiply the $i^\text{th}$ coordinate by
    $\beta^{-1}$.
  \item [$c_i \gets c_i + \beta c_j$] Change the $j^\text{th}$ coordinate
    to $\alpha_j - \beta \alpha_i$.
  \end{description}
  It is easy to see that these are isomorphisms.
\end{proof}

\begin{theorem}
  $\trank(T) + \tnullity(T) = \tdim(T)$
\end{theorem}
\begin{proof}
  Let $A$ be a matrix that corresponds to $T$. Let $A'$ be the CREF of
  $A$ with $r = \trank(T)$ non-zero columns. The null-space of $A'$
  has dimension $n - r$ which is the same as the dimension of the
  null-space of $A$.
\end{proof}

\Closesolutionfile{ans}
\section*{Answers}
\input{ans1}

\end{document}
